\documentclass[a4paper]{scrartcl}
% преамбула
\usepackage{polyglossia}
\setdefaultlanguage[forceheadingpunctuation=false]{russian}
\setotherlanguage{english}
\defaultfontfeatures{Ligatures=TeX}

\usepackage{hyperref, xcolor}
\usepackage{cite}
\hypersetup{backref, colorlinks=true, linkcolor=black, urlcolor=black, linktocpage, pdfborder = {0 0 1 [1 0]}}

%Указание используемых шрифтов
\setmainfont{Times New Roman}
\newfontfamily{\cyrillicfont}{Times New Roman}
\setsansfont{Arial}
\newfontfamily{\cyrillicfontsf}{Arial}
\setmonofont{Courier New}
\newfontfamily{\cyrillicfonttt}{Courier New}

%Вставка в документ растровых изображения
\usepackage{graphics, graphicx}

%Пакеты для работы с таблицами
\usepackage{longtable, ltcaption}
\usepackage{array, tabu, tabularx, tabulary}

\usepackage[top=20mm, bottom=20mm, left=30mm, right=10mm]{geometry}

\KOMAoptions{fontsize=14}
\setlength{\parindent}{12.5mm}
\usepackage[onehalfspacing]{setspace}
\setkomafont{section}{\bfseries\rmfamily}
\RedeclareSectionCommand[indent=12.5mm]{section}

%подписи таблицы и картинки
\usepackage{caption}
\captionsetup[table]{format=default, labelsep=endash}
\captionsetup[figure]{labelsep=endash}
\captionsetup[table]{labelsep=endash, singlelinecheck = off, position = above}
\renewcaptionname{russian}{\figurename}{Рисунок}
\renewcaptionname{russian}{\contentsname}{\begin{center}СОДЕРЖАНИЕ\end{center}}

%Формулы
\usepackage[fleqn]{amsmath}
\setlength{\parindent}{1.25cm}% абзацный отступ
%\setlength{\mathindent}{12.5mm}

\usepackage{hyperref}

\usepackage{placeins}

\usepackage{indentfirst}
\usepackage{tocloft}
\renewcommand{\cftsecleader}{\cftdotfill{\cftsubsecdotsep}} % Отточие для разделов
\setlength{\cftbeforesecskip}{0.5em} % Расстояние между строками разделов
\setlength{\cftsecindent}{0pt} % Разделы без отступов
\setlength{\cftsubsecindent}{2em} % Подразделы с отступом 2 знака
\setlength{\cftsubsubsecindent}{4em} % Пункты с отступом 4 знака
\renewcommand{\cfttoctitlefont}{\bfseries\rmfamily} % Times New Roman, нормальный стиль, центр

\begin{document}

%\tableofcontents{}

\begin{titlepage}
\newgeometry{top=10mm, bottom=10mm, left=10mm, right=10mm}
\begin{center}
{\large Министерство науки и высшего образования РФ\\
 Федеральное государственное автономное\\
 образовательное учреждение высшего образования}\\
«\textbf{{\large СИБИРСКИЙ ФЕДЕРАЛЬНЫЙ УНИВЕРСИТЕТ}}»\\[5mm]
Институт космических и информационных технологий\\
кафедра «Системы искусственного интеллекта»\\
\vfill
\textbf{\large ПРАКТИЧЕСКАЯ РАБОТА}\\[5mm]
по дисциплине «Введение в профессиональную деятельность»\\
«Оформление документа учебной деятельности»\\
\end{center}
\vfill
\begin{tabular}{l c m{0.01\textwidth} l}
Преподаватель&\rule{3cm}{0.1pt}& & А.~В.~Кушнаренко\\[0mm]
&{\footnotesize подпись, дата}& & \\[10mm]
Студент группы VAR_GROUP № з/к VAR_CARD &\rule{3cm}{0.1pt}& & VAR_STUDENT_SIGNATURE\\[1mm]
&{\footnotesize подпись, дата}& & \\
\end{tabular}
\vfill
\begin{center}
Красноярск {\the\year}
\end{center}
\end{titlepage}

\restoregeometry
\tableofcontents

\newpage
\section{Введение}

\section{Основная часть}

\section{Вывод}

\end{document}